\chapter{Introduction}
\label{chap:introduction}

\section{Background}
Co-curricular activities play a pivotal role in the holistic development of university students. Clubs and organizations within universities provide platforms for students to explore their interests, develop leadership skills, and engage in community service. In Bangladesh, many public and private universities have vibrant club cultures, ranging from science and technology clubs to debating societies and cultural groups. However, a significant gap exists in the digital presence of these organizations. While the activities are robust, the documentation and showcasing of these achievements often lack professional presentation.

The Science Club of Mawlana Bhashani Science \& Technology University (MBSTU) identified this need. To elevate the club's profile to a national and international standard, a dedicated digital platform was required. This platform would not only serve as a portfolio but also as an operational tool to manage members, events, and, crucially, the certification process which is often cumbersome and manual.

\section{Problem Statement}
Most university clubs in Bangladesh face several challenges regarding operations and branding:
\begin{enumerate}
    \item \textbf{Lack of Portfolio:} Clubs often do not have a centralized website to showcase their history, executive committees, and achievements.
    \item \textbf{Manual Certification:} Issuing certificates for competitions or memberships is a manual process involving design, printing, and physical signing, which is time-consuming and prone to errors.
    \item \textbf{Verification Issues:} Physically issued certificates are hard to verify by third parties (e.g., employers or other universities), reducing their perceived value.
    \item \textbf{Data Management:} Member records are often kept in spreadsheets, leading to data loss or redundancy.
\end{enumerate}

This project aims to solve these problems by introducing a Software as a Service (SaaS) web application tailored for the Science Club.

\section{Objectives}
The primary objectives of this project are:
\begin{itemize}
    \item To develop a dynamic website that showcases the Science Club's vision, mission, and activities.
    \item To automate the certificate generation process with a custom template designer and digital signature integration.
    \item To create a secure verification system where certificates can be validated online.
    \item To provide an administrative panel for managing team members, events, galleries, and FAQs.
    \item To implement a secure advisor panel where faculty advisors can approve certificates remotely.
\end{itemize}

\section{Scope of the Project}
The scope of the Science Club Management System includes:
\begin{itemize}
    \item \textbf{Public Portal:} A responsive website accessible to visitors for viewing club details, galleries, and verifying certificates.
    \item \textbf{Admin Portal:} A secured dashboard for the Executive Committee to manage content, design certificate templates, and handle applications.
    \item \textbf{Advisor Portal:} A dedicated interface for university professors (advisors) to review and approve certificate requests via secure email links.
    \item \textbf{Certificate Engine:} A canvas-based tool for designing templates and an automated PDF/Image generator for issuing certificates.
\end{itemize}

\section{Report Organization}
The remainder of this report is organized as follows:
\begin{itemize}
    \item \textbf{Chapter 2} provides a review of relevant literature and technologies.
    \item \textbf{Chapter 3} details the system analysis and requirements.
    \item \textbf{Chapter 4} outlines the system design and architecture.
    \item \textbf{Chapter 5} discusses the implementation details and results.
    \item \textbf{Chapter 6} concludes the report with future recommendations.
\end{itemize}
