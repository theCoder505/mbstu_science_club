\section{Introduction}
This chapter provides a deep dive into the actual code implementation of the system. We will examine the key Controllers, Middlewares, and Frontend logic that drive the application \cite{clean_code}.

\section{Middleware Implementation}
Middlewares act as the gatekeepers of the application.

\subsection{Sharing Global Data (Inertia Middleware)}
In `HandleInertiaRequests.php`, we define what data is available to every Vue component automatically.

\begin{lstlisting}[language=PHP, caption=HandleInertiaRequests.php]
class HandleInertiaRequests extends Middleware
{
    public function share(Request $request): array
    {
        // Fetch Global Settings
        $settings = \App\Models\WebSetting::first();
        
        return [
            ...parent::share($request),
            // Global App Info
            'name' => config('app.name'),
            
            // User Auth State
            'auth' => [
                'user' => $request->user(),
                'advisor' => Auth::guard('advisor')->user(),
            ],
            
            // Sidebar State (Persisted in Cookie)
            'sidebarOpen' => ! $request->hasCookie('sidebar_state') || $request->cookie('sidebar_state') === 'true',
            
            // Flash Messages for SweetAlert
            'flash' => [
                'success' => $request->session()->get('success'),
                'error' => $request->session()->get('error'),
            ],
            
            // Data from Settings Table (Footer Info)
            'webSettings' => [
                'contact_email' => $settings->contact_email ?? 'info@example.com',
                'phone_number' => $settings->phone_number ?? '+8801xxx',
                'facebook_link' => $settings->facebook_link ?? '#',
                // ... other social links
            ]
        ];
    }
}
\end{lstlisting}

\subsection{Advisor Access Control}
To ensure Advisors can't access Admin routes and vice-versa.

\begin{lstlisting}[language=PHP, caption=AdvisorMiddleware.php]
public function handle(Request $request, Closure $next): Response
{
    // Check if the user is logged in via 'advisor' guard
    if (!Auth::guard('advisor')->check()) {
        return redirect()
            ->route('advisor_panel.login_page')
            ->with('error', 'Please login to continue');
    }

    return $next($request);
}
\end{lstlisting}

\section{Controller Logic}

\subsection{Certificate Generation (ApplicationController)}
This is the logic that processes the application verification.

\begin{lstlisting}[language=PHP, caption=ApplicationController.php (Snippet)]
public function validate_application(Request $request, $id) {
    $application = Apply::findOrFail($id);
    
    // Check if member dates are valid
    if(Carbon::parse($application->member_till)->isPast()) {
        return back()->with('warning', 'Membership has expired.');
    }
    
    $application->status = 'verified';
    $application->verified_at = now();
    $application->save();

    // Trigger Email Notification to Advisor
    $advisor = Advisor::where('department', $application->department)->first();
    if($advisor) {
        Mail::to($advisor->email)->send(new ReviewApplicationMail($application));
    }

    return redirect()->route('applications')->with('success', 'Verified & Sent to Advisor');
}
\end{lstlisting}

\section{Frontend Implementation (Vue.js)}

\subsection{OTP Input Component}
A reusable component for the OTP verification.

\begin{lstlisting}[language=JavaScript, caption=OtpInput.vue]
<script setup>
import { ref, watch } from 'vue';

const props = defineProps(['modelValue']);
const emit = defineEmits(['update:modelValue']);

const digits = ref(['', '', '', '', '', '']);

const handleInput = (index, event) => {
    const val = event.target.value;
    if (val && index < 5) {
        // Auto-focus next input
        document.getElementById(`otp-${index + 1}`).focus();
    }
    emitCode();
};

const emitCode = () => {
    emit('update:modelValue', digits.value.join(''));
};
</script>

<template>
  <div class="flex gap-2">
    <input v-for="(digit, i) in 6" 
           :key="i"
           :id="`otp-${i}`"
           v-model="digits[i]"
           maxlength="1"
           class="w-12 h-12 text-center border rounded-lg focus:ring-2 ring-indigo-500"
           @input="handleInput(i, $event)" />
  </div>
</template>
\end{lstlisting}

\section{Asset Compilation (Vite)}
We utilize Vite for lightning-fast HMR (Hot Module Replacement) during development \cite{vite}.

\begin{lstlisting}[language=JavaScript, caption=vite.config.ts]
import { defineConfig } from 'vite';
import laravel from 'laravel-vite-plugin';
import vue from '@vitejs/plugin-vue';

export default defineConfig({
    plugins: [
        laravel({
            input: 'resources/js/app.ts',
            ssr: 'resources/js/ssr.ts',
            refresh: true,
        }),
        vue({
            template: {
                transformAssetUrls: {
                    base: null,
                    includeAbsolute: false,
                },
            },
        }),
    ],
});
\end{lstlisting}
