\section{Conclusion}
The successful development and deployment of the Club Portfolio Management System (CPMS) marks a pivotal moment in the operational history of the MBSTU Science Club. This project has demonstrated that "Digital Transformation" is not just a buzzword but a tangible upgrade that can simplify lives, even for student organizations.

By leveraging the power of Modern SaaS Architecture (Laravel 12, Vue 3, Inertia), we have created a system that is:
\begin{enumerate}
    \item \textbf{Secure:} Uses industry-standard encryption and guard-based auth.
    \item \textbf{Scalable:} Ready for future growth.
    \item \textbf{User-Centric:} Solves real pain points like certificate issuance.
\end{enumerate}

\section{Future Roadmap}
While the current version is robust, several avenues for improvement identified during the "Literature Review" remain to be implemented:

\subsection{Phase 2: Payment Integration}
We plan to integrate \textbf{SSLCommerz} to allow for:
\begin{itemize}
    \item Automated membership fee collection.
    \item Selling merchandise (T-Shirts) directly from the site.
\end{itemize}

\subsection{Phase 3: Multi-Tenancy}
The ultimate goal is to offer this software to \textit{other} clubs in MBSTU. By adding a \texttt{club\_id} column to every table and using Laravel's Tenant package, we can host 50+ clubs on a single server, creating a university-wide ecosystem.

\section{Final Thoughts}
This project has been an extensive learning journey in Full Stack Development. It bridged the gap between academic theory (ERD, Normalization) and industry practice (Deployments, CI/CD, Unit Testing). The CPMS stands ready to serve the next generation of scientists at MBSTU.
